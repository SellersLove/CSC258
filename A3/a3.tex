\RequirePackage[l2tabu,orthodox]{nag}  % warn about common LaTeX pitfalls
\RequirePackage[ascii]{inputenc}  % input is 7-bit ASCII
\RequirePackage{fixltx2e}  % fix LaTeX2e kernel bugs

\documentclass[11pt,twoside]{article}
\usepackage{color}
\usepackage{graphicx}
\graphicspath{ {image/} }
\usepackage{calc}  % arithmetic in length parameters
\usepackage{enumitem}  % more control over list formatting
\usepackage{fancyhdr}  % simpler headers and footers
\usepackage[margin=1in]{geometry}  % page layout
\usepackage{lastpage}  % for last page number
\usepackage{relsize}  % easier font size changes
\usepackage[normalem]{ulem}  % smarter underlining
\usepackage{url}  % verb-like typesetting of URLs
\usepackage{xfrac}  % nicer looking simple fractions for text and math
\usepackage{longtable}
\usepackage{tikz}
\usepackage{array}
\usepackage{tikz-timing}
\usetikzlibrary{arrows, shapes, backgrounds,fit}
\usepackage{tkz-graph}
% Set up fonts.
\usepackage[T1]{fontenc}  % use true 8-bit fonts
\usepackage{slantsc}  % allow slanted small-caps
\usepackage{microtype}  % perform various font optimizations
% Use Palatino-based monospace instead of kpfonts' default.
%\usepackage{newpxtext}
\ttfamily
\DeclareFontShape{T1}{\ttdefault}{m}{scsl}{<->ssub*\ttdefault/m/sc}{}
\DeclareFontShape{T1}{\ttdefault}{b}{scsl}{<->ssub*\ttdefault/b/sc}{}
% "Kepler" fonts.
\usepackage[nott,notextcomp]{kpfonts}
% Use curvier Latin Modern brackets instead of kpfonts' glyphs.
\DeclareSymbolFont{lmsymb}     {OMS}{lmsy}{m}{n}
\DeclareSymbolFont{lmlargesymb}{OMX}{lmex}{m}{n}
\DeclareMathDelimiter{\rbrace}{\mathclose}{lmsymb}{"67}{lmlargesymb}{"09}
\DeclareMathDelimiter{\lbrace}{\mathopen}{lmsymb}{"66}{lmlargesymb}{"08}

% Page layout: stretch text to fill up page.
\addtolength\footskip{.25\headheight}
\flushbottom

% Common list settings.

% Common macros.
\input{macros-263}
\newcommand*\st{\mathrel{|}}  % "such that" for set extension

% Headings.
\pagestyle{fancy}
\let\headrule\empty
\let\footrule\empty
\lhead{CSC\,258\,H1}
\chead{\large\scshape Assignment \#\,3}
\rhead{\scshape Fall 2015}
\lfoot{\scshape Dept.\@ of Computer Science, University of Toronto,
       St.~George Campus}
\cfoot{}
\rfoot{\scshape page \thepage\space of \pageref{LastPage}}


\begin{document}

\begin{enumerate}[leftmargin=0pt]

\item The basic idea for the code, is first read the input character by character, then do the arithmetic, finally print the number with 4 digits precision. 
\begin{itemize}[label={}]
	\item When reading the input, we keep tracking whether the input is a operator (i.e \ $+-*/ .$) or a number.
	\item If it is a number, then we track if it is the fractional part or integer part. In order to do so we create a boolean variable $fraction$. If we encounter a dot we set  $fraction$ to true (note: we use $1$ for true $-1$ for false).
	We also have a boolean variable called negative in order to track whether the next operand is negative
	or not. 
	\item If the input is an operator $+ - */$, we check whether the previous input is a number or dot. If it is a number or a dot then all operators are valid and we do the corresponding operation. If it is not a number then only $-$ is valid, which means next operand is negative. Otherwise, ignore the input and brunch to read input.
	\item If the input is $=$ then we check if the previous input is a number or dot. If it is then $=$ is a valid input, otherwise ignore the input and brunch to read input. After we received $=$ then the only valid input would be $esc$, and if the next input is $esc$, we brunch to print result otherwise, ignore and wait for the next input.
	If we received a $esc$ input and the previous input is not $=$, then just ignore and wait for next input.
	\item For printing part, we are basically using the idea from the course website for \proc{printint}, with a slight modification. For printing a floating number we just keep tracing the digits of the integer part and decimal position. 
	\item After we print the result we brunch back to the beginning of the program and waiting for the next computation. 
	
\end{itemize}
	\textcolor{red}{code and sample output are attached with more detailed comment}  
	\[ \includegraphics[scale=0.35]{output.png} \]
\item
	\begin{enumerate}
	%%(a)
	\item BNT $m$
		\begin{align} \nonumber
                		AC\ \ \  &\rightarrow \ \ \ AL1\\ \nonumber
                		one \ \ \ & \rightarrow \ \ \ AL2 \\ \nonumber
                		IR \ \ \ & \rightarrow \ \ \ ALF\\ \nonumber
                		ALU \ \ \ & \rightarrow \ \ \ AC\\ \nonumber
                		IR \ \ \ & \rightarrow \ \ \ m0\\ \nonumber
                		PC+1 \ \ \ &  \rightarrow \ \ \ m1\\ \nonumber
                		AC \ \ \ & \rightarrow \ \ \ MS\\ \nonumber
                		MUX \ \ \ & \rightarrow \ \ \ PC \nonumber
		\end{align}
	%% (b)
	\item EXE $m$	
		\begin{align} \nonumber
                		IR\ \ \ &\rightarrow  \ \ \ MAR\\ \nonumber
                		RAM \ \ \ &\rightarrow \ \ \ IR \nonumber
		\end{align}
	
	\end{enumerate}

\end{enumerate}

\end{document}