\RequirePackage[l2tabu,orthodox]{nag}  % warn about common LaTeX pitfalls
\RequirePackage[ascii]{inputenc}  % input is 7-bit ASCII
\RequirePackage{fixltx2e}  % fix LaTeX2e kernel bugs

\documentclass[11pt,twoside]{article}
\usepackage{color}
\usepackage{graphicx}
\graphicspath{ {image/} }
\usepackage{calc}  % arithmetic in length parameters
\usepackage{enumitem}  % more control over list formatting
\usepackage{fancyhdr}  % simpler headers and footers
\usepackage[margin=1in]{geometry}  % page layout
\usepackage{lastpage}  % for last page number
\usepackage{relsize}  % easier font size changes
\usepackage[normalem]{ulem}  % smarter underlining
\usepackage{url}  % verb-like typesetting of URLs
\usepackage{xfrac}  % nicer looking simple fractions for text and math
\usepackage{longtable}
\usepackage{tikz}
\usepackage{array}
\usepackage{tikz-timing}
\usetikzlibrary{arrows, shapes, backgrounds,fit}
\usepackage{tkz-graph}
% Set up fonts.
\usepackage[T1]{fontenc}  % use true 8-bit fonts
\usepackage{slantsc}  % allow slanted small-caps
\usepackage{microtype}  % perform various font optimizations
% Use Palatino-based monospace instead of kpfonts' default.
%\usepackage{newpxtext}
\ttfamily
\DeclareFontShape{T1}{\ttdefault}{m}{scsl}{<->ssub*\ttdefault/m/sc}{}
\DeclareFontShape{T1}{\ttdefault}{b}{scsl}{<->ssub*\ttdefault/b/sc}{}
% "Kepler" fonts.
\usepackage[nott,notextcomp]{kpfonts}
% Use curvier Latin Modern brackets instead of kpfonts' glyphs.
\DeclareSymbolFont{lmsymb}     {OMS}{lmsy}{m}{n}
\DeclareSymbolFont{lmlargesymb}{OMX}{lmex}{m}{n}
\DeclareMathDelimiter{\rbrace}{\mathclose}{lmsymb}{"67}{lmlargesymb}{"09}
\DeclareMathDelimiter{\lbrace}{\mathopen}{lmsymb}{"66}{lmlargesymb}{"08}

% Page layout: stretch text to fill up page.
\addtolength\footskip{.25\headheight}
\flushbottom

% Common list settings.

% Common macros.
\input{macros-263}
\newcommand*\st{\mathrel{|}}  % "such that" for set extension

% Headings.
\pagestyle{fancy}
\let\headrule\empty
\let\footrule\empty
\lhead{CSC\,258\,H1}
\chead{\large\scshape Assignment \#\,2}
\rhead{\scshape Fall 2015}
\lfoot{\scshape Dept.\@ of Computer Science, University of Toronto,
       St.~George Campus}
\cfoot{}
\rfoot{\scshape page \thepage\space of \pageref{LastPage}}


\begin{document}

\begin{enumerate}[leftmargin=0pt]
% question 1 
\item 
	\begin{enumerate}
	% part (a)
	\item 
		$a = \bot$ and $b = \bot$,
		\begin{itemize}[label = {}]
		\item Set $x = \bot$ and $y = \bot$, the output is still $x = \bot$ and $y = \bot$; 
		hence, the circuit stays stable.
 		\item  Set $x = \bot$ and $y = \top$, then the output chnage to $x = \bot$ and $y = \bot$;
		hence, unstable.
		\item Set $x = \top$ and $y = \bot$, then the output changes to $x = \top$ and $y = \top$; 
		hence, unstable.
		\item Set $x = \top$ and $y = \top$, the output is still $x = \top$ and $y = \top$; 
		hence, the circuit stays stable.
		\end{itemize}
	The stable states are $(a,b,x,y) = (\bot, \bot, \bot, \bot)$ or $(a,b,x,y) = (\bot, \bot, \top, \top)$. 
	%part (b)
	\item 
		For stable state $(a,b,x,y) = (\bot, \bot, \bot, \bot)$, when you apply a pulse,
		\begin{enumerate}
		\item apply pulse to $a$:
			\[\begin{tikztimingtable} 
			 a	& 10L 10H 10L \\
			 b	& 30L \\
			 x 	& 30L\\
			 y 	&30L\\
			\end{tikztimingtable} \]
		\item apply pulse to $b$:
			\[\begin{tikztimingtable} 
			 a	& 30L \\
			 b	& 10L 10H 10L \\
			 x 	& 10L 10H 10L \\
			 y 	& 10L 10H 10L \\
			\end{tikztimingtable} \]
		\item apply pulse to both $a$ and $b$:
			\begin{itemize}[label = {}]
			\item if $a$ happens first:
                			\[\begin{tikztimingtable} 
                			 a	& 10L 10H 10L \\
                			 b	& 12L 10H 8L \\
                			 x 	& 20L 10H\\
                			 y 	& 12L 18H \\
                			\end{tikztimingtable} \] 
			 \item if $b$ happens first:
			        \[\begin{tikztimingtable} 
                			 a	& 10L 10H 10L \\
                			 b	& 8L 10H 12L \\
                			 x 	& 8L 2H 20L\\
                			 y 	& 8L 10H 12L\\
                			\end{tikztimingtable} \] 
			\item if $a$ and $b$ happen at same time:
			        \[\begin{tikztimingtable} 
                			 a	& 10L 10H 10L \\
                			 b	& 10L 10H 10L \\
                			 x 	& 20L2H2L2H2L2H \\
                			 y 	& 10L 10H 2H2L2H2L2H \\
                			\end{tikztimingtable} \] 				
			\end{itemize}
		\end{enumerate}
		For stable state $(a,b,x,y) = (\bot, \bot, \top, \top)$, when you apply a pulse,
		\begin{enumerate}
		\item apply pulse to $a$:
			\[\begin{tikztimingtable} 
			 a	& 10L 10H 10L \\
			 b	& 30L \\
			 x 	& 10H 20L\\
			 y 	&10H 20L\\
			\end{tikztimingtable} \]
		\item apply pulse to $b$:
			\[\begin{tikztimingtable} 
			 a	& 30L \\
			 b	& 10L 10H 10L \\
			 x 	& 30H \\
			 y 	& 30H \\
			\end{tikztimingtable} \]
		\item apply pulse to both $a$ and $b$:
			\begin{itemize}[label = {}]
			\item if $a$ happens first:
                			\[\begin{tikztimingtable} 
                			 a	& 10L 10H 10L \\
                			 b	& 12L 10H 8L \\
                			 x 	& 10H 10L 10H\\
                			 y 	& 10H 2L 18H \\
                			\end{tikztimingtable} \] 
			 \item if $b$ happens first:
			        \[\begin{tikztimingtable} 
                			 a	& 10L 10H 10L \\
                			 b	& 8L 10H 12L \\
                			 x 	& 10L 20L\\
                			 y 	& 18H 12L\\
                			\end{tikztimingtable} \] 
			\item if $a$ and $b$ happen at same time:
			        \[\begin{tikztimingtable} 
                			 a	& 10L 10H 10L \\
                			 b	& 10L 10H 10L \\
                			 x 	& 10H 10L 2H2L2H2L2H \\
                			 y 	& 22H 2L2H2L2H \\
                			\end{tikztimingtable} \] 				
			\end{itemize}
		\end{enumerate}
	\end{enumerate}
% question 2
\item 
	We know that the $JK latchs$
        \begin{longtable}{|>{\tiny}p{0.1in}|>{\tiny} p{0.1in}| >{\tiny}p{0.2in}|}
        \hline
       	J&K&Q\\[0.1in]\hline
	$\bot$&$\bot$&$\triangleleft Q$\\[0.1in]\hline
	$\bot$&$\top$&$\bot$\\[0.1in]\hline
	$\top$&$\bot$&$\top$\\[0.1in]\hline
	$\top$&$\top$&$-\triangleleft Q$\\[0.1in]\hline
	\end{longtable} 
	Also, it is clearly to see that: 
	\[ J_A = -\triangleleft Q_C;  \ K_A = -\triangleleft Q_C\]
	\[ J_B = \triangleleft Q_A;  \ K_B = \triangleleft Q_A\]
	\[ J_C = \triangleleft Q_A \wedge \triangleleft Q_B;  \ K_C = \triangleleft Q_C\]
	from the tables above we can easily come up with the diagram:
			\[\begin{tikztimingtable} 
                			 clock	& 2L 2H 2L 2H 2L 2H 2L 2H 2L 2H 2L 2H 2L 2H 2L 2H 2L\\
                			 $Q_A$	& 6H 4L 4H 8L 4H 4L 4H \\
                			 $Q_B$	& 2H 4L 8H 12L 8H  \\
                			 $Q_C$ 	& 2H 12L 4H 16L  \\
                		\end{tikztimingtable} \] 

%question 3
\item
Here we show the circuit for $n=4$:
		\[ \includegraphics[scale=0.5]{q3.png} \]
%question 4
\item 
	\begin{enumerate}
	\item
		convert -42 to IEEE single-precision floating point.
		\begin{itemize}[label = {}]
		\item -42 in binary is $-101010$,
		\[\begin{tabular}{c@{\,}c@{\,}c@{\,}c}
  				& 4 &2  &  \\
			        & 2& 1&  \ \ \ 0\\
  				& 1 &0  & \ \ \ 1 \\
			        &    &  5&  \ \ \ 0\\
				&    &  2& \ \ \ 1 \\
				&    &  1& \ \ \ 0 \\
				&    &  0& \ \ \ 1 \\
		\end{tabular}\]
		 which is $-1.01010 \times 2^5 = -1.01010 \times 2^{132-127}$;
		\item similarly we get 132 in binary is $10000100$;
		\item IEEE single-precision standard: $[1 \ 10000100 \ 01010 00000 00000 00000 000]$.
		\end{itemize}
	\item 
		convert 3.14 to IEEE single-precision floating point.
		\begin{itemize}[label = {}]
		\item 
		\[\begin{tabular}{c@{\,}c@{\,}c@{\,}c}
  				0.&  1&4  &  \\
			        0.& 2& 8&  \ \ \ 0\\
  				0.& 5&6  & \ \ \ 0\\
				1.& 1&2  & \ \ \ 1 \\
				 0.& 2& 4&  \ \ \ 0\\
				  0.& 4& 8&  \ \ \ 0\\
				   0.& 9& 6&  \ \ \ 0\\
				  1.& 9& 2&  \ \ \ 1\\
				  1.& 8& 4&  \ \ \ 1\\
				  1.& 6& 8&  \ \ \ 1\\
				  1.&3& 6&  \ \ \ 1\\
				   0. &7& 2&  \ \ \ 0\\
				 1. &4& 4&  \ \ \ 1\\
				  0. &8& 8&  \ \ \ 0\\
				  1. &7& 6&  \ \ \ 1\\
				  1. &5& 2&  \ \ \ 1\\
				 1. &0& 4&  \ \ \ 1\\
				 0. &0& 8&  \ \ \ 0\\
				 0. &1& 6&  \ \ \ 0\\
				0. &3& 2&  \ \ \ 0\\
				0. &6& 4&  \ \ \ 0\\
				1. &2& 8&  \ \ \ 1\\
					\end{tabular}\]
		\item  $3$ in binary is $11$, and $0.14$ in binary is $0.0 \ \overline{01000111101011100001}$
		\item $3.14$ in binary is $11.0 \ \overline{01000111101011100001} = 1.10 \ \overline{01000111101011100001} \times 2^1$;
		\item normalize $314_2 = 1.10 \ \overline{01000111101011100001} \times 2^{128-127} $;
		\item 128 in binary is $10000000$;
		\item IEEE single-precision standard: $[0 \ 10000000 \ 10 01000 11110 10111 00001 0 ]$.
		\end{itemize}
	\end{enumerate}
%question 5
\item  convert $[0 \ 10000011 \ 01011 01011 01011 01011 010]$ from  IEEE single-precision floating-point to base ten.
	\begin{itemize}[label = {}]
	\item from the first bit $0$ we can see that the number is positive;
	\item the next 8 bits $10000011$ stand for exponent which is $131$, then the exponent is $131-127 = 4$;
	\item then the next stand for significant, which is $1.35483860$
	\item the number is $1.35483860 \times 2^4 = 21.677$ with 5 significant digits.
	\end{itemize}
%question 6
\item
\begin{enumerate}
\item In order to calculated $-\frac{5}{6}$, we first compute $\frac{1}{3}$, then $\frac{5}{3}$, then $\frac{5}{6}$ and finally $-\frac{5}{6}$
\begin{itemize}[label = {}]
	\item  compute $\frac{1}{3}$ is $1 \div 11 = 01'1$:
		\[\begin{tabular}{c@{\,}c@{\,}c@{\,}c}
  				&  &  & 1' \\
			-       &  & 1 & 1 \\
				\hline
  				&  &\  1' &  \\
			-       &  1  & 1 &  \\
				\hline
				1' &   0  &  &  \\
		\end{tabular}\]
	\item compute then $\frac{5}{3}$ is $01'1 \times 101 = 10'111$:
			\[\begin{tabular}{c@{\,}c@{\,}c@{\,}c@{\,}c}
  				&&  0&  \ 1'& 1 \\
			+     01' &1& 1 &  &  \\
				\hline
  				&10'\ \ \ &  1&\   1& 1 \\
			\end{tabular}\]
	\item compute then $\frac{5}{6}$ by shifting the decimal point:
					\[10'11.1\]
	\item compute $\frac{-5}{6}$ by 2's complement:
					\[01'00.1\]
	
	\end{itemize}
\item We know that $7$ is binary is $111$; hence, int binary quote notation is $111$ as well.
\item compute $01'00.1 -111 = 10'0 1101. 1$:
			\[\begin{tabular}{c@{\,}c@{\,}c@{\,}c@{\,}c}
		            -    &01'&  0& 0.& 1 \\
		            	& 0'1 &  1&1&  \\
				\hline
  				10'0 & 11 &  0&1.&  1\\
			\end{tabular}\]
\item generate from the quote notation paper we have the formula: \\ 
		 let
		 \[x = 10\]
		 \[y = 01101.1\]
		 \[a = 1\]
		 \[w = sequence \ of \ a's \ the \ same \ length\ as\  x\]
		 \[z = digit \ 1\  followed\ by \ a \ sequence \ of\ zeros\ of\ the\ same\ length\ as\ y\]
		then
		\[(10'0 1101. 1)_{10} =y- \frac{xz}{w} = 01101.1 - \frac{10 \times 1 00000}{11}\]
		\[ = 13\frac{1}{2} - \frac{64}{3}\]
		\[ = \frac{81}{6} - \frac{128}{3} = -\frac{47}{6} = -7.8\bar3\]
\end{enumerate}

%question 7
\item
\begin{itemize}[label = {}]
	\item First convert decimal ASII to binary, and we have:
	\item $d$ in ASCII code is $100$ in decimal, then $1100100$ in binary;
	\item $o$ in ASCII code is $111$ in decimal, then $1101111$  in binary;
	\item $space$ in ASCII code is $32$ in decimal,then $0100000$  in binary;
	\item $i$ in ASCII code is $105$ in decimal, then $1101001$  in binary;
	\item $t$ in ASCII code is $116$ in decimal,then $1110100$  in binary;
	\item Then using Hamming code to encode we have,
        \begin{longtable}{|>{\tiny}p{0.3in}|>{\tiny} p{0.1in}| >{\tiny}p{0.1in}|>{\tiny}p{0.1in}|>{\tiny} p{0.1in}|>{\tiny}p{0.1in}
        |>{\tiny}p{0.1in}|>{\tiny}p{0.1in}|>{\tiny}p{0.1in}|>{\tiny}p{0.1in}|>{\tiny}p{0.1in}|>{\tiny}p{0.1in}|>{\tiny}p{0.1in}}
        \hline
       $symbol$&$p_1$&$p_2$&$d_1$&$p_3$&$d_2$&$d_3$&$d_4$&$p_4$&$d_5$&$d_6$&$d_7$\\[0.1in]\hline
	$b	     $&$1    $&$1    $&$1    $&$1    $&$1    $&$0    $&$0   $&$1    $&$1    $&$0    $&$0    $\\[0.1in]\hline
	$o	     $&$1    $&$0    $&$1    $&$0    $&$1    $&$0    $&$1   $&$1    $&$1    $&$1    $&$1    $\\[0.1in]\hline
	$space  $&$1    $&$0    $&$0    $&$1    $&$1    $&$0    $&$0   $&$0    $&$0    $&$0    $&$0    $\\[0.1in]\hline
	$i	     $&$0    $&$1    $&$1    $&$0    $&$1    $&$0    $&$1   $&$1    $&$0    $&$0    $&$1    $\\[0.1in]\hline
	$t	     $&$1    $&$0    $&$1    $&$0    $&$1    $&$1    $&$0   $&$1    $&$1    $&$0    $&$0    $\\[0.1in]\hline
	\end{longtable}
	\item 
	\item If the third bit of third character is flipped. Then we will get ${10} 1 {1} 100 {0} 000$ for the third character.
	\item We know that : 
		\[c_1 = p_1d_1d_2d_4d_5d_7\]
		\[c_2 = p_2d_1d_3d_4d_6d_7\]
		\[c_3 = p_3d_2d_3d_4\]
		\[c_4 = p_4d_5d_6d_7\]
	\item Then, $c_1 = 1$, $c_2 = 1$, $c_3 = 0$ and $c_4 = 0$. Hence, we get $0011$, then we know the third bit is flipped.
\end{itemize}
\end{enumerate}

\end{document}